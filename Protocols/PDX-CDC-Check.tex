% Options for packages loaded elsewhere
\PassOptionsToPackage{unicode}{hyperref}
\PassOptionsToPackage{hyphens}{url}
%
\documentclass[
]{article}
\usepackage{amsmath,amssymb}
\usepackage{iftex}
\ifPDFTeX
  \usepackage[T1]{fontenc}
  \usepackage[utf8]{inputenc}
  \usepackage{textcomp} % provide euro and other symbols
\else % if luatex or xetex
  \usepackage{unicode-math} % this also loads fontspec
  \defaultfontfeatures{Scale=MatchLowercase}
  \defaultfontfeatures[\rmfamily]{Ligatures=TeX,Scale=1}
\fi
\usepackage{lmodern}
\ifPDFTeX\else
  % xetex/luatex font selection
\fi
% Use upquote if available, for straight quotes in verbatim environments
\IfFileExists{upquote.sty}{\usepackage{upquote}}{}
\IfFileExists{microtype.sty}{% use microtype if available
  \usepackage[]{microtype}
  \UseMicrotypeSet[protrusion]{basicmath} % disable protrusion for tt fonts
}{}
\makeatletter
\@ifundefined{KOMAClassName}{% if non-KOMA class
  \IfFileExists{parskip.sty}{%
    \usepackage{parskip}
  }{% else
    \setlength{\parindent}{0pt}
    \setlength{\parskip}{6pt plus 2pt minus 1pt}}
}{% if KOMA class
  \KOMAoptions{parskip=half}}
\makeatother
\usepackage{xcolor}
\usepackage[margin=1in]{geometry}
\usepackage{longtable,booktabs,array}
\usepackage{calc} % for calculating minipage widths
% Correct order of tables after \paragraph or \subparagraph
\usepackage{etoolbox}
\makeatletter
\patchcmd\longtable{\par}{\if@noskipsec\mbox{}\fi\par}{}{}
\makeatother
% Allow footnotes in longtable head/foot
\IfFileExists{footnotehyper.sty}{\usepackage{footnotehyper}}{\usepackage{footnote}}
\makesavenoteenv{longtable}
\usepackage{graphicx}
\makeatletter
\def\maxwidth{\ifdim\Gin@nat@width>\linewidth\linewidth\else\Gin@nat@width\fi}
\def\maxheight{\ifdim\Gin@nat@height>\textheight\textheight\else\Gin@nat@height\fi}
\makeatother
% Scale images if necessary, so that they will not overflow the page
% margins by default, and it is still possible to overwrite the defaults
% using explicit options in \includegraphics[width, height, ...]{}
\setkeys{Gin}{width=\maxwidth,height=\maxheight,keepaspectratio}
% Set default figure placement to htbp
\makeatletter
\def\fps@figure{htbp}
\makeatother
\setlength{\emergencystretch}{3em} % prevent overfull lines
\providecommand{\tightlist}{%
  \setlength{\itemsep}{0pt}\setlength{\parskip}{0pt}}
\setcounter{secnumdepth}{-\maxdimen} % remove section numbering
\ifLuaTeX
  \usepackage{selnolig}  % disable illegal ligatures
\fi
\IfFileExists{bookmark.sty}{\usepackage{bookmark}}{\usepackage{hyperref}}
\IfFileExists{xurl.sty}{\usepackage{xurl}}{} % add URL line breaks if available
\urlstyle{same}
\hypersetup{
  pdftitle={RTX Resistance EXP1 Plan},
  pdfauthor={Callum Malcolm},
  hidelinks,
  pdfcreator={LaTeX via pandoc}}

\title{RTX Resistance EXP1 Plan}
\author{Callum Malcolm}
\date{2024-03-07}

\begin{document}
\maketitle

\hypertarget{overview}{%
\section{Overview}\label{overview}}

\begin{itemize}
\tightlist
\item
  To establish cell lysis capabilities of RTX on BL PDX N2 and RBL2P
\end{itemize}

Protocol adapted from:

CDC Assays

CDC studies were performed as previously described {[}9{]}. In brief,
2×10°6 CLL cells in 1mL AIM-V medium were pretreated on ice for 30
minutes with 10μg/mL of each mAb either alone or in combination. These
cell suspensions were then split and either received no serum, 10\%
(v/v) NHS, or 10\% (v/v) C5-serum. The cells were then incubated for one
hour at 37°C with 95\% humidity and 5\% CO2 and then washed twice with
3mL of AIM-V medium before analysis. Absolute viable cell counts were
assayed using flow cytometry with BD Trucount beads (BD Biosciences) in
a 1\% BSA buffer with PI staining to assess cell viability. Cells killed
by CDC can either be lysed (i.e.~disintegrated and no longer detected on
flow cytometry) or remain intact with PI permeable membranes (intact
dead cells). Cell lysis was determined for each sample by dividing the
absolute cell count of the experimental specimens by that of the control
specimen (10\%NHS). Total CDC (\% cytotoxicity) which measures the sum
of lysed cells and intact dead cells was determined for each sample by
dividing the absolute viable (PI negative) counts of the experimental
specimens by that of the control specimen followed by multiplication by
100.

\href{https://www.sciencedirect.com/science/article/pii/S0145212605003607?via\%3Dihub}{Flow
Analysis of RTX-resistant B-NHL cell lines}\\
RAMOS cells adjusted to 1 × 106 cells/ml were incubated both with and
without rituximab in the presence of human serum for 1 h at 37 °C, and
samples were cooled for 5 min at 0 °C to stop the reaction. The
percentage of lysed cells was determined immediately by direct cell
count in trypan blue-stained samples, as previously described. Cell
lysis was examined and expressed as a percentage as follows: \%cell
lysis = \{1 − (viable cells after exposure to rituximab and human
serum)/(viable cells before exposure to rituximab and human serum)\}
×100.

\hypertarget{experimental-plan}{%
\section{Experimental Plan}\label{experimental-plan}}

\hypertarget{day-1}{%
\subsection{Day 1}\label{day-1}}

\begin{itemize}
\tightlist
\item
  Thaw PDX onto feeders in T175 flask
\end{itemize}

\hypertarget{day-2}{%
\subsection{Day 2}\label{day-2}}

\begin{itemize}
\tightlist
\item
  Check cells and replenish media if needed
\end{itemize}

\hypertarget{day-3}{%
\subsection{Day 3}\label{day-3}}

\begin{longtable}[]{@{}ccc@{}}
\toprule\noalign{}
Sample Number & EBV Status & CD20 \\
\midrule\noalign{}
\endhead
\bottomrule\noalign{}
\endlastfoot
7 & 5 Negative & 7 Positive \\
& 1 Positive & \\
& 1 Not tested & \\
\end{longtable}

\end{document}
